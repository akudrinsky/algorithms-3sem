\begin{task}{362}
    В выпуклом многограннике все грани являются правильными 5-, 6- или 7-угольниками. Докажите, что пятиугольных граней ровно на 12 больше, чем семиугольных.
\end{task}

\begin{solution}
    Обозначим за $k_{i}$ количество $i$-угольников среди граней. Разобьем наше решение на шаги.
    \begin{enumerate}
        \item Посчитаем количество ребер $E$ в многограннике. Каждое ребро есть пересечение двух граней. Значит, $2 \cdot E = 5 \cdot k_{5} + 6 \cdot k_{6} + 7 \cdot k_{7}$.
        \item Посчитаем количество граней: $G = k_{5} + k_{6} + k_{7}$.
        \item Посчитаем количество вершин многогранника. Вершина в $\mathcal{R}^3$ может лежать в пересечении не менее чем трех плоскостей. Докажем, что в нашем случае каждая вершина лежит на пересечении ровно трех граней. Пусть это не так, и вершина $A$ лежит на пересечении $n>3$ граней. Заметим, что любой угол правильного $n$-угольника равен $\frac{(n - 2) \cdot 180^{\circ}}{n}$. Тогда сумма углов многоугольников при вершине $A$ как минимум $4 \cdot \frac{(5 - 2) \cdot 180^{\circ}}{5} > 360^{\circ}$, что невозможно. Полученное противоречие показывает, что каждая вершина лежит на пересечении трех граней. Значит, $3 \cdot V = 5 \cdot k_{5} + 6 \cdot k_{6} + 7 \cdot k_{7}$.
        \item Далее заметим, что выпуклый многоугольник это планарный граф (достаточно <<растянуть>> одну из граней, после чего спроецировать все на плоскость). Таким образом, $V - E + G = 2$. Выразив найденные величины, получим, что $\frac{5 \cdot k_{5} + 6 \cdot k_{6} + 7 \cdot k_{7}}{3} - \frac{5 \cdot k_{5} + 6 \cdot k_{6} + 7 \cdot k_{7}}{2} + k_{5} + k_{6} + k_{7} = 2$ \Rightarrow $k_{5} - k_{7} = 12$.
    \end{enumerate}
    Итак, утверждение доказано.
\end{solution}
    